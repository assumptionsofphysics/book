\chapter{Classical mechanics}

The standard view in physics is that classical mechanics is perfectly understood. It has three different equivalent formulations, the oldest of which, Newtonian mechanics, is based on three laws. Classical mechanics is the theory of point particles that follow that laws. Unfortunately, this view is incorrect.

We will see that the three formulations are not equivalent, in the sense that there are physical systems that are Newtonian but not Hamiltonian and vice-versa. There are also a number of questions that are left unanswered, such as the precise nature of the Hamiltonian or the Lagrangian, and what exactly the principle of stationary actions represents physically. While shedding light on these issues, we will also find that classical mechanics already contains elements that are typically associated with other theories, such quantum mechanics (uncertainty principle, anti-particles), thermodynamics/statistical mechanics (thermodynamic and information entropy conservation) or special relativity (energy as the time-component of a four-vector). In other words, the common understanding of classical mechanics is quite shallow, and its foundations are not separate to the ones of classical statistical mechanics and special relativity.

What reverse physics shows is that the central assumption underneath classical mechanics is that of \textbf{infinitesimal reducibility}: a classical system can be thought as made of parts, which in terms are made of parts and so on; studying the whole system is equivalent to studying all its infinitesimal parts. This assumptions, together with the assumptions of \textbf{independence of degrees of freedom}, is what gives us the structure of classical phase space with conjugate variables. The additional assumption of \textbf{determinism and reversibility}, the fact that the description of the system at one time is enough to predict its future or reconstruct its past, leads us to Hamiltonian mechanics. Conversely, assuming \textbf{kinematic equivalence}, the idea that trajectories in space are enough to reconstruct the state of the system and vice-versa, leads us to Newtonian mechanics. The combination of all assumption, instead, leads to Lagrangian mechanics and, in particular, to massive particles under (scalar and vector) potential forces.

\section{Formulations of classical mechanics}

In this section we will briefly review the three main formulations of classical mechanics. Our task is not to present them in detail, but rather to provide a brief summary of the equations so that we can proceed with the comparison. Note that different conventions are used across formulations, within the same formulation and among different contexts (e.g. relativity, symplectic geometry). Given that we need to span all these different cases, and given that the different notations clash with each other, we will make a synthesis that will hopefully feel familiar enough.

\subsection{Newtonian mechanics}

For all formulations, the system is modeled as a collection of point particles, though we will mostly focus on the single particle case. For a Newtonian system, the state of the system at a particular time $t$ is described by the position $x^i$ and velocity $v^i$ of all its constituents. Each particle has its mass $m$, not necessarily constant in time, and, for each particle, we define kinetic momentum as $\Pi^i = m v^i$.\footnote{We will use the letter $t$ for the time variable, $x$ for position and $v$ for velocity, which is a very common notation in Newtonian mechanics. However, we will keep using the same letters in Lagrangian mechanics as well, instead of $q$ and $\dot{q}$, for consistency. Given that the distinction between kinetic and conjugate momentum is an important one, we will note $\Pi$ the first and $p$ the second. The Roman letters $i,j,k,...$ will be used to span the spatial components, while we will use the Greek letters $\alpha, \beta, \gamma, ...$ to span space-time components. Unlike some texts, $x^i$ do not represent Cartesian coordinates, and therefore they should be understood already as generalized coordinates.}

The evolution of our system is given by the Newton's second law:
\begin{equation}\label{rp-cm-NewtonsSecondLaw}
	F^i(x^j, v^k, t) = \frac{d \Pi^i}{dt}.
\end{equation}
Mathematically, if the forces $F^i$ are Lipschitz continuous, then the solution is unique. That is, given position and velocity at a given time, we can predict the position and velocity at future times. We will assume a system Newtonian has this property.

An important aspect of Newtonian mechanics is that the equations are not invariant under coordinate transformation. To distinguish between apparent forces (i.e. those dependent on the choice of frame) and the real ones, we assume the existence of inertial frames. In an inertial frame there are no apparent forces, and therefore a free system (i.e. no forces) with constant mass proceeds in a linear uniform motion, or stays still.

\subsection{Lagrangian mechanics}

The state for a Lagrangian system is also given by position $x^i$ and velocity $v^i$. The dynamics is specified by a single function $L(x^i, v^j, t)$ called Lagrangian. For each spatial trajectory $x^i(t)$ we define the action as $\mathcal{A}[x^i(t)] = \int_{t_0}^{t_1} L(x(t), d_t x(t), t) dt$.\footnote{For derivatives, we will use the shorthand $d_t$ for $\frac{d}{dt}$ and $\partial_{x^i}$ for $\frac{\partial}{\partial x^i}$. } The trajectory taken by the system is the one that makes the action stationary:
\begin{equation}
\delta \mathcal{A}[x^i(t)] = \delta \int_{t_0}^{t_1} L\left(x(t), d_t x(t), t\right) dt=0
\end{equation}
The evolution can alternatively be specified by the Euler-Lagrange equations:
\begin{equation}\label{rp-cm-EulerLagrange}
	\partial_{x^i}L=d_t \partial_{v^i} L.
\end{equation}

Note that not all Lagrangians lead to a unique solution. For example, $L=0$ will give the same action for all trajectories and therefore, strictly speaking, all trajectories are possible. The stationary action leads to a unique solution if and only if the Lagrangian is hyperregular, which means the Hessian matrix $\partial_{v^i}\partial_{v^j} L$ is invertible. Like in the Newtonian case, we will assume Lagrangian systems satisfy this property.

Unlike Newton's second law, both the Lagrangian and the Euler-Lagrange equations are invariant under coordinate transformations. This means that Lagrangian mechanics is particularly suited to study the symmetries of the system.

\subsection{Hamiltonian mechanics}

In Hamiltonian mechanics, the state of the system is given by position $q^i$ and conjugate momentum $p_i$. The dynamics is specified by a single function $H(q^i, p_j, t)$ called Hamiltonian.\footnote{We use a different symbol for position in Hamiltonian mechanics because, while it is true that $q^i = x^i$, it is also true that $\partial_{q^i} \neq \partial_{x^i}$: the first derivative is taken at constant conjugate momentum while the second is taken at constant velocity. This creates absolute confusion when mixing and comparing Lagrangian and Hamiltonian concepts, which our notation completely avoids.} The evolution is given by Hamilton's equations:
\begin{equation}
	\begin{aligned}
		d_t q^i = \partial_{p_i} H \\
		d_t p_i = - \partial_{q^i} H \\
	\end{aligned}
\end{equation}

TODO Conditions for uniqueness. Hamiltonian must be at least differentiable.

Hamilton's equations are also invariant. The Hamiltonian itself is a scalar function which is often considered (mistakenly as we'll see later) invariant. This formulation is the most suitable for statistical mechanics as volumes of phase space correctly count the number of possible configurations.

\section{Inequivalence of formulations}

It is often stated in physics books that all three formulations of classical mechanics are equivalence. We will look at this claim in detail, and conclude that this is not the case: there are systems that can be described by one formulation and not another. More precisely, the set of Lagrangian systems is exactly the intersection of Newtonian and Hamiltonian systems.

We will consider two formalism equivalent if they can be applied exactly to the same systems. That is, Newtonian and Lagrangian mechanics are equivalent if any system that can be described using Newtonian mechanics can also be described by Lagrangian mechanics and vice-verse. In general, in physics great emphasis is put on systems that can indeed be studied by all three, leaving the impression that this always doable.\footnote{If one asks the average physicist whether Newtonian and Hamiltonian mechanics, the answer will be most of the time enthusiastically positive. If one then asks to provide the Hamiltonian for a damped harmonic oscillator, the typical reaction is annoyance to the nonsensical question (damped harmonic oscillator do not conserve energy), followed by a realization and partial retraction of the previous claim. The moral of the story is to never take these claims at face value.} However, just with a cursory glance, we realize that this can't possibly be the case.

The dynamics of a Newtonian system, in fact, is specified by three independently chosen functions of position and velocity, the forces applied to each degree of freedom. On the other hand, the dynamics of Lagrangian and Hamiltonian systems are specified by a single function of position and velocity/momentum, the Lagrangian the Hamiltonian. Intuitively, there are more choices in the dynamics for Newtonian systems than for Lagrangian and Hamiltonian.

Now, the reality is a bit trickier because the mathematical expression of the forces is not enough to fully characterize the physical system. We need to know in which frame we are, what coordinates are being used and the mass of the system, which is potentially a function of time. On the Lagrangian side, note that the Euler-Lagrange equations are homogeneous in $L$. This means that multiplying $L$ by a constant leads to the same solutions, meaning that the same system can be described by more than one Lagrangian. The converse is also true: if one system is half as massive and is subjected to a force field half as intense, the resulting Lagrangian is also simply rescaled by a constant factor. Therefore the map between Lagrangians and Lagrangian system is not one-to-one: is many-to-may. This is why we should never look simply at mathematical structures if we want to fully understand the physics.

Regardless, our task is at the moment much simpler: we only need to show that there are Newtonian systems not expressible by Lagrangian or Hamiltonian mechanics. We can therefore limit ourselves to systems with a specific constant mass $m$ in an inertial frame. Every possible expression of the force is allowed and will lead to a unique expression of the acceleration $a^i=F^i(x^i, v^i, t)/m$ which means a unique set of possible trajectories for each expression of the force. Now, we can think of trying to write a Lagrangian for each of those systems. Assuming that this is possible, the acceleration $a^i=F^i[L]/m$ is going to be some functional of that Lagrangian, which, given the Euler-Lagrange equations \ref{rp-cm-EulerLagrange} must be continuous: for small variation of the Lagrangian we must have a small variation of the equations of motion and therefore of the acceleration. But a surjective continuous map from the space of a single function (i.e. the Lagrangian) and the space of multiple functions (i.e. those that specify the forces) does not exist, and therefore there must be at least one Netwonian system with constant mass expressed in an inertial frame that is not describable using Lagrangian mechanics. The same argument applies for Hamiltonian mechanics, since the dynamics in this case is also described by a single function in the same number of arguments. We therefore reach the following conclusion:
\begin{equation}
	\textrm{Not all Netwonian systems are Lagrangian and/or Hamiltonian.}
\end{equation}

We now want to understand whether all Lagrangian systems are Newtonian. Given what we discussed, we cannot expect to reconstruct the mass and force uniquely from the expression of the Lagrangian. We consider the mass and the frame fixed by the problem, together with the Lagrangian, and therefore we must only see whether we can indeed find a unique expression for the acceleration. From the Euler-Lagrange equations \ref{rp-cm-EulerLagrange} we can write
\begin{equation}
	\begin{aligned}
	\partial_{x^i}L&=d_t \partial_{v^i} L=\partial_{q^j} \partial_{v^i} L \, d_t q^j + \partial_{v^k} \partial_{v^i} L \, d_t v^k = \partial_{q^j} \partial_{v^i} L \, v^j + \partial_{v^k} \partial_{v^i} L \, a^k \\
	\partial_{v^k} &\partial_{v^i} L \, a^k = \partial_{x^i}L - \partial_{q^j} \partial_{v^i} L \, v^j .
	\end{aligned}
\end{equation}
To be able to write the acceleration explicitly, we must be able to invert the Hessian matrix $\partial_{v^k} \partial_{v^i} L$. As we said before, this is exactly the condition for which the principle of stationary actions leads to a unique solution, and we can better understand why. If it is not invertible at a point, the determinant is zero and therefore one eigenvalue is zero. The corresponding eigenvector correspond to a direction for which the equation tells us nothing, and therefore a variation of the acceleration in that direction will not change the action. This is why the invertibility of the Hessian is required to obtain unique solutions.

What we find, then, is that for any Lagrangian system, which we assume to have a unique solution, we can explicitly write the acceleration as a function of position, velocity and time. Therefore
\begin{equation}
	\textrm{All Lagrangian systems are Newtonian.}
\end{equation}

Now we turn our attention to Hamiltonian mechanics and, similarly, we ask whether we can express the acceleration as a function of position and velocity. We have
\begin{equation}
	\begin{aligned}
		a^i &= d_t v^i = d_t d_t q^i = d_t \partial_{p_i} H = \partial_{q^j} \partial_{p_i} H d_t q^j + \partial_{p_k} \partial_{p_i} H d_t p_k \\
		&= \partial_{q^j} \partial_{p_i} H \partial_{p_j} H - \partial_{p_k} \partial_{p_i} H \partial_{q^k} H.
	\end{aligned}
\end{equation}
This tells us that the acceleration is always an explicit function, but it is, in general, an explicit function of position and momentum, not of position and velocity. To do that, we need to be able to express the momentum as a function of position and velocity. Note that Hamilton's equations already give a way to express the velocity in terms of position and momentum, we just need that expression to be invertible, which means the Jacobian must be invertible. We have:
\begin{equation}
	\left|\partial_{p_i} v^j\right| = \left|\partial_{p_i}\partial_{p_j} H\right| \neq 0 .
\end{equation}
To be able to express momentum as a function of position and velocity, then, we need the Hessian of the Hamiltonian to be invertible, to have non-zero determinant.

Note that we had no requirement for the Hamiltonian. For example, $H=0$ leads to $d_t q^i = 0$ and $d_t p_i = 0$, which both position and momentum are constants of motion. The Hessian, being the zero matrix, is not invertible, and in fact we cannot write momentum as a function of position and velocity: velocity is always zero in all cases. Though this case may not be physically interesting, it leads to unique solutions: it is a perfectly valid Hamiltonian system. While it is instructive to always check the trivial mathematical case, let us go through a more physically meaningful case.

\textbf{Photon as a particle}. If we want to treat the photon as a classical particle, we can write the Hamiltonian by expressing the energy as a function of momentum
\begin{equation}
	H=\hbar | \omega| = c \hbar |k| = c |p|.
\end{equation}
If we apply Hamilton's equations, we have
\begin{equation}
	\begin{aligned}
		d_t q^i &= c \frac{p^i}{|p|} \\
		d_t p_i &= 0.
	\end{aligned}
\end{equation}
That is, the norm of the velocity is always $c$ and the momentum decides the direction, and the momentum does not change in time. This is indeed the motion of a free photon. One can confirm, through tedious calculation, that the determinant of the Hessian is indeed zero, yet it is easier and more physically instructive to see that we cannot reconstruct the momentum from the velocity. Relativistically, all photons travel along the geodesics at the same speed, therefore two photons that differ only by the magnitude of the momentum will travel the same path.

Let us give a name to this extra condition.
\renewcommand{\theassump}{KE}%
\begin{assump}[Kinematic Equivalence]\label{assum_kineq}
	The kinematics of the system is enough to reconstruct its dynamics and vice-versa.
\end{assump}
\renewcommand{\theassump}{\Roman{assump}}%
By kinematics we means the motion in space and time and by dynamics we mean the state and its evolution in phase space. We will need to analyze the difference between the two more in detail, but we should first finish our comparison between the different formulations.

Summing up, we find that
\begin{equation}
	\textrm{Not all Hamiltonian systems are Newtonian: only those for which  \ref{assum_kineq} is valid.}
\end{equation}


We now need to compare Lagrangian and Hamiltonian systems. The task is a lot easier because we already have a precise way to connect the two. If we are given a Lagrangian $L$, we define the conjugate momentum $p_i = \partial_{v^i} L$ and the Hamiltonian $H = p_i v^i - L$. If we are given a Hamiltonian $H$, we can define a Lagrangian $L = p_i v^i - H$. The only detail that need to be understood is whether this can be done for all Lagrangian and Hamiltonian systems.

While these expression are always define, what needs to be checked is whether we can change variables; whether we can write the Lagrangian in terms of position and velocity and the Hamiltonian in terms of position and momentum. Going from a Hamiltonian to a Lagrangian, it again means that we can write momentum as a function of velocity, and therefore assumptions \ref{assum_kineq}. This makes sense: if all Lagrangian systems are Newtonian, and \ref{assum_kineq} was required for a Hamiltonian system to be Newtonian, then it also required for a Hamiltonian system to be Lagrangian. But the connection is stronger: \ref{assum_kineq} is the only assumption we need to be able to write a Lagrangian from a Hamiltonian.

Going from a Lagrangian to a Hamiltonian, it means that we can write velocity as a function of position and momentum. Note that since we define conjugate momentum as the derivative of the Lagrangian, we can already express momentum as a function of position and velocity, which means we are simply asking that expression to be invertible. This is, again, assumption \ref{assum_kineq}, just in the opposite direction. We must have
\begin{equation}
	0 \neq \left| \partial_{v^i} p_j \right| = \left| \partial_{v_i} \partial_{v_j} L \right|.
\end{equation}
This means that assumption \ref{assum_kineq} is exactly the invertibility of the Hessian, the condition for unique solution of the Lagrangian. In fact, we can write
\begin{equation}
	\left| \partial_{v_i} \partial_{v_j} L \right| = \left| \partial_{v^i} p_j \right| = \left| \partial_{p_i} v^j \right|^{-1} = \left|\partial_{p_i}\partial_{p_j} H\right|^{-1}.
\end{equation}

This means that every Lagrangian admits a Hamiltonian, but not every Hamiltonian admits a Lagrangian. Only the Hamiltonian systems for which \ref{assum_kineq} is valid will also be Lagrangian system, with a guaranteed unique solution given that \ref{assum_kineq} is exactly the assumption needed for that as well. Therefore we conclude that
\begin{equation}
	\textrm{Lagrangian systems are exactly those Hamiltonian systems for which \ref{assum_kineq} is valid.}
\end{equation}

The relationship between the different formulations, then, can be summarized with the following Venn diagram.

TODO: Venn diagram of the different formulations

Lagrangian mechanics is exactly the intersection of Newtonian mechanics and Lagrangian mechanics.

We have found that \ref{assum_kineq} is a constitutive assumption of Lagrangian mechanics, and that it clearly marks which Hamiltonian systems are Newtonian/Lagrangian. By constitutive assumption we mean an assumption that must be taken, either explicitly or implicitly, for a theory to be valid. But what makes a system Hamiltonian and what makes a system Newtonian? Can we find a full set of constitutive assumptions for classical mechanics?

\section{Kinematics vs dynamics}

