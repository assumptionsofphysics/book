
\def\>{\rangle}
\def\<{\langle}

%\DeclareMathOperator{\mix}{mix}
%\DeclareMathOperator{\component}{comp}
%\DeclareMathOperator{\cospan}{cospan}
\newcommand\mix{\mathrm{mix}}
\newcommand\component{\mathrm{comp}}
\newcommand\cospan{\mathrm{cospan}}

\newcommand{\ens}[1][e] {\mathsf{#1}} % Ensemble
\newcommand{\Ens}[1][E] {\mathcal{#1}} % Ensemble space

\chapter{Ensemble spaces}

Here we summarize the work in progress for a general theory of states and processes that is applicable to any physical system. The tentative core concept is the ensemble, therefore we want to find definitional requirements for ensembles and then find suitable assumptions to recover the different theories (e.g. classical, quantum and thermodynamic).

 Conceptually, we want to argue that physical theories are, and can only be, about ensembles. At a practical level, most of the time we can only prepare and measure statistical properties as we do not have infinite control over all systems. The cases where properties can be prepared with one hundred percent reliability can still be understood as a ensembles of identical preparations, though one can argue for an ensemble of external conditions or internal dynamics. At a conceptual level, the goal of physics is to write laws that apply all the time: every time that one prepares a system according to a particular procedure, let's it evolve in particular conditions, he will obtain a particular result. That is, the idea of repeatability of experimental results implicitly assumes that the objects of scientific inquiry are not single instances, but the infinite collections of all reproducible instances.

\section{Working examples}

A general theory of ensembles will need to be able to recover classical spaces and quantum spaces under appropriate physically meaningful assumptions. There are two aspects that are common to all cases, that can be suitably generalized: statistical mixing and entropy. It is useful, then, to review the cases we need to recover. For now, we will leave field theories out of this treatment.

\subsection{Classical case - discrete}

\subsubsection{Convex structure}

The space of classical distributions over a discrete space corresponds to a simplex (\url{https://en.wikipedia.org/wiki/Simplex}). In the finite case, the pure states $\mathcal{G} = \{g_i\}_{i=1}^{n} \subset \Ens$ are finitely many and each ensemble $e = \sum p_i g_i$ is uniquely identified by a decomposition of pure states. Effectively, each ensemble is a probability distribution over the pure states. Mathematically, each point of the space is a convex combination of the vertices. The simplex has a center point, which corresponds to the maximally mixed state, a uniform distribution over all pure states.

The countably infinite case is a bit different because not all infinite sequences $\{p_i\} \in [0,1]$ converge. As we cannot create a uniform distribution over infinitely many cases, there is no center point. Effectively, there is ``hole'' in the middle. More precisely, the space is not topologically closed in the sense it does not contain all the limit points.

The uncountably infinite case is not physically relevant (i.e. no second countable discrete topology, cases are not experimentally decidable).

TODO: Understand the space of the limits. There should be at least one point for each sequence $\{p_i\}$ that converges to a finite $p < 1$. Intuitively, we can keep that part of the distribution constant while we spread the rest uniformly to all cases. Each should reach a different limit point.

\subsubsection{Entropic structure}

Each ensemble $e$ has a single decomposition $e = \sum p_i g_i$ in terms of pure states. The entropy is given by $-\sum p_i \log p_i$ for both finite and countable cases. Therefore the entropy for each pure state is assumed to be zero. The entropy of the maximally mixed state is $\log n$ where $n$ is the number of pure states. The entropy increase as we go from pure states to the maximally mixed state. The level sets (i.e. the fibers) of the entropy form a series of concentric ``shells''.

Note that the entropy is not finite for all convergent $p_i$ (i.e. infinite convex combinations). For details, see \url{https://arxiv.org/pdf/1212.5630.pdf}. Given that we will want the entropy to exist and be finite for all ensembles, this means that simply generalizing the convex space to all probability measures (like in \url{https://ncatlab.org/nlab/show/superconvex+space} does not seem physically warranted.

The fact that all pure states have the same entropy is an additionally hypothesis that cannot work in general. To see this, consider the case where the state is defined by the number of molecules for two substances. This space is the product of two independent variable $n_a$ and $n_b$. If we have a uniform distribution over $N_a$ cases of $n_a$ and $N_b$ cases of $n_b$, the total number of cases is $N_a N_b$. Therefore the entropy of the joint state is the sum of the entropy of the marginals. However, if we pair $n_a$ with the total number of molecules $n_{(a+b)}$ we have a problem. The issue that variable $n_{(a+b)}$ corresponds to a variable number of joint cases.

\subsection{Classical case - continuous}

\subsubsection{Convex structure}

In the continuous case, the space of ensembles is the space of integrable functions over a symplectic manifold (e.g.  over phase space) that integrate to one. That is, if $X$ is a symplectic manifold, then $\Ens = L^1(X, \mu)$ where $\mu=\int \omega^n$ is the Liouville measure. The pure states are the space of delta distributions $\mathcal{G} = \{\delta_x\}_{x \in X}$, but they are not in  $\Ens$: they are limit points. If we imagine a uniform distribution over a region of the symplectic manifold, we now have two limits that are not part of the ensemble space: the limit with infinite support, and the limit of zero support. In some sense, a distribution can be understood as a convex integral of the zero support limits: $\rho(x) = \int_X \rho(y) \delta_x(y) d\mu$. Even in this case, each ensemble can be understood as a probability distribution (density in this case) over all pure states. The symplectic nature of the manifold is require to assign a frame invariant density to states and a frame invariant notion of independence between DOFs.

TODO: We have already shown that that the ability to define frame-independent probability densities on pure states is linked to the existence symplectic structure. The notion of independence of DOF should be connected to a notion of product of convex spaces. As defined below, the entropic structure is defined already in a frame independent way (at least at constant time). There may be a way to make the argument tighter from a mathematical perspective.

TODO: The pure states are not part of the space: they are limits. Technically, they are not extreme points. We should have a definition of pure states that recovers all notions of pure states.

\subsubsection{Entropic structure}

The entropy is given by $- \int \rho \log \rho d\mu$ where $\mu$ is the Liouville measure and $\rho$ is the probability density over canonical coordinates. If a different measure is used, or if the coordinates are not canonical, the formula gives the wrong result.

Unlike the classical discrete case, subspaces and dimensionality of subspaces cannot be defined without the entropy. The issue is that we need a measure on the set of ``extreme points'', and the convex structure cannot provide it. The entropy, however, does as the supremum of the entropy for all distributions with support $U$ is $\log \mu(U)$.

The entropy can be used to both identify subspaces and recover the Liouville measure. Take $\rho \in X$; construct the set $\text{dis}(\rho)$ of all the ensembles that are disjoint from $\rho$; now construct the set $\text{con}(\rho)$ of all ensembles that are disjoint from all elements of $\text{dis}(\rho)$. These will be the set of ensembles that share the supports. Find the supremum of $\text{dis}(\rho)$ in terms of the entropy. If it doesn't exists, the support is infinite. If it does, the support is finite and the measure is $2^{\sup S(\text{con}(\rho))}$.

TODO: Study the shell of zero entropy states. For example, it should not be path connected: all uniform distributions with support of the same finite size (in terms of Liouville measure) will have the same entropy. The region, however, need not be contiguous. Since we cannot continuously transform a single region into two disjoint regions, there will be different distributions at zero entropy that cannot be transformed continuously.


\subsection{Quantum case - finite dimensional}

\subsubsection{Convex structure}

For a qubit, the Bloch ball is the space of ensembles $\Ens$. That is, let $\mathcal{H}$ be a two dimensional Hilbert space. The interior of the block sphere correspond to mixtures (i.e. the space of positive semi-definite Hermitian operators with trace one $M(\mathcal{H})$) while the surface corresponds to the pure states $\mathcal{G}$ (i.e. $\{ |\psi\> \<\psi| \}_{\psi \in \mathcal{H}}$). In this case, there is no single decomposition. The general finite dimensional case works similarly. Note that the space is exactly characterized by knowing which different mixtures provide the same ensemble.

The multiple decompositions make the ensemble space behave in a way that is a hybrid between the classical discrete and continuous. Pure states are properly a part of the state, as in the discrete case, and we can describe each mixture in terms of finitely many states. However, the pure state are a continuum, therefore we can also define probability densities over the space, convex integrals. For example, for a single qubit, the maximally mixed state (the center of the sphere) can be equally described as the equal mixture of two opposite states (e.g. spin up and spin down, or spin left and spin right). However, it can also be described as the equal mixture of the whole sphere.

Note that complex projective are symplectic, which is what allows to define frame invariant densities. Ideally, whatever formal argument we are able to work for classical mechanics would also apply to quantum mechanics.  Also note that the two dimensional sphere is the only symplectic sphere. By homogeneity, we should be able to argue that the space is symmetric around the maximally mixed state, and is therefore a ball. The symplectic requirement would select dimension two. Note that real and quaternionic spaces would be excluded by this argument.

\subsubsection{Entropic structure}

In quantum mechanics, the entropy is the von Neumann entropy for both finite and infinite dimension. The entropy for the maximally mixed state is $\log n$ where $n$ is the dimensionality of the Hilbert space. Note that, to calculate the von Neumann entropy, we are diagonalizing the density matrix $\rho$. This means finding a set of orthogonal pure states $g_i$ such that $\rho = \sum p_i g_i$ is a convex combination. Note that the convex hull of a set of $n$ orthogonal pure states is an $n$-dimensional simplex whose center is the maximally mixed state. Therefore, we are looking for the smallest simplex that contain $\rho$ and the maximally mixed state. In the two dimensional case, $\rho$ is an interior point. Take the line that connects $\rho$ to the center. The two points of the sphere are the extreme points for the decomposition. The distance from the points will be proportional to the probability. Because of this property, the von Neumann entropy is the smallest Shannon entropy between all possible decompositions.


\subsection{Quantum case - infinite dimensional}

The infinite dimensional can be both understood as the countably infinite dimensional version of the discrete case, or as the space of complex square integrable functions. As in the  (i.e. $L^2$) should work the same, except that there is no upper bound on the entropy, and therefore there is no maximally mixed state.

As in the classical infinite case, the maximally mixed state is not in the convex space.

TODO: It is not clear what is the correct space for the infinite dimensional case. The Hilbert space contains pure states that have infinite expectation values. Schwartz spaces do not have this problem, but the order of the basis seems relevant as the rate of convergence is defined with that in mind. A reorder of the basis, in fact, can change the rate of convergence. Take, for example, a series that converges faster than any polynomial. Now construct a series by adding a zero term in between every non-zero term. The series will also converge faster than any polynomial. Now rearrange the zero so that the terms get more and more spread out. At some point, the rate of convergence of the series will change. It is unclear what is the relationship between the topology of these limits points and the convex/entropic structure. Related question is whether $S(\mathbb{R})$ and $S(\mathbb{R}^2)$ isomorphic as extreme points of a convex set.


\section{Ensemble spaces, convex spaces and convex sets}

Basic definitions for convex spaces are taken from \url{https://ncatlab.org/nlab/show/convex+space} and \url{https://arxiv.org/abs/0903.5522}. Note that we are not interested in convex spaces per se, but their use in modeling ensembles. Therefore, as is customary in the physical mathematics approach of Assumptions of Physics, we are going to be using a terminology that is specific to the problem at hand. However, nothing is settled about the terminology.

\begin{defn}
	For a real number $p \in [0,1]$, $\bar{p} = 1-p$.
\end{defn}

\begin{defn}
	An \textbf{ensemble} represents infinite copies, taken at once, of the output of a process. An \textbf{ensemble space} is a collection of ensembles that describe the same system under all processes. Formally, an ensemble space $\Ens$ is a set of elements, called ensembles, equipped with an operation $+ : [0, 1] \times \Ens \times \Ens \to \Ens$ called \textbf{mixing}, typically noted with the infix notation $p \ens_1 + \bar{p} \ens_2$, with the following properties:
	\begin{itemize}
		\item \textbf{Identity}: $1 \ens_1 + 0 \ens_2 = \ens_1$
		\item \textbf{Idempotence}:  $p \ens_1 + \bar{p} \ens_1 = \ens_1$ for all $p \in [0,1]$
		\item \textbf{Commutativity}: $p \ens_1 + \bar{p} \ens_2 = \bar{p} \ens_2 + p \ens_1$ for all $p \in [0,1]$
		\item \textbf{Associativity}: $p_1 \ens_1 + \bar{p_1}\left(\overline{\left(\frac{p_3}{\bar{p_1}}\right)}\ens_2 + \frac{p_3}{\bar{p_1}}\ens_3\right) =  \bar{p_3}\left(\frac{p_1}{\bar{p_3}} \ens_1 +  \overline{\left(\frac{p_1}{\bar{p_3}}\right)}\ens_2\right) + p_3 \ens_3$ for all $p_1, p_3 \in [0,1]$
	\end{itemize}
	Given symmetry and associativity, we can write $p_1 \ens_1 + p_2 \ens_2 + p_3 \ens_3$ where $p_2 = \bar{p_1}\overline{\left(\frac{p_3}{\bar{p_1}}\right)} = \bar{p_3}\overline{\left(\frac{p_1}{\bar{p_3}}\right)} = 1 - p_1 - p_3$. We can also extend to \textbf{convex combinations} $\sum p_i \ens_i$ where $\sum_i p_i = 1$.
\end{defn}
\begin{justification}
	Let $\ens_1$ and $\ens_2$ be the ensembles that represent the output of two different processes $P_1$ and $P_2$. Let a selector $S_p$ be a process that outputs two symbols, the first with probability $p$ and the second with probability $\bar{p}$. Then we can create another process $P$ that, depending on the selector, outputs either the output of $P_1$ and $P_2$. Therefore we are justifying in assuming the mixing operation among ensembles.
	
	If $p=1$, the output of $P$ will always be the output of $P_1$. This justifies the identity. If $P_1$ and $P_2$ are the same process, then, again, the output of $P$ will always be the output of $P_1$. This justifies the idempotence. As long as the same probability is matched to the same output, the process $P$ is identical. This justifies the symmetry. If we are mixing three processes $P_1$, $P_2$ and $P_3$, as long as the final probabilities are the same, it does not matter if we mix $P_1$ and $P_2$ first or $P_2$ and $P_1$. This justifies associativity.
\end{justification}

\begin{remark}
	All classical and quantum cases fit this definition.
\end{remark}

\begin{coro}
	An ensemble space is a convex space.
\end{coro}

\begin{defn}
	Let $\Ens$ be an ensemble space. Let $\ens[\rho] = \sum_i p_i \ens_i$ where $\ens[\rho], \{\ens_i\} \in \Ens$ and $p_i \in [0,1]$ such that $\sum p_i = 1$. We say that $\ens[\rho]$ is a \textbf{mixture} of $\{\ens_i\}$ and each $\ens_i$ is a \textbf{component} of $\ens[\rho]$. Let $\ens[\rho] = p \ens_1 + \bar{p} \ens_2$. Then we say that $\ens_2$ is a $p$-\textbf{complement} of $\ens_1$ towards $\ens[\rho]$.
\end{defn}

\begin{remark}
	While most of the literature on convex spaces focus on ``combinations'', the inverse ``decomposition'' is crucial. For example, whether a convex space is a vector space depends on properties of decomposition.
\end{remark}

\begin{defn}
	An ensemble space is \textbf{complemented} if all $p$-complements are unique.
\end{defn}

\begin{prop}
	An ensemble space embeds into a real vector space with convex combination as linear combination if and only if it is complemented.
\end{prop}
\begin{proof}
	Theorem 4 in \url{https://arxiv.org/abs/1105.1270} states that a convex space embeds into a real vector space with $c_\lambda(x,y) = \lambda x + \bar{\lambda}y$ if and only if
	$$ c_\lambda(x,y) = c_\lambda(x,z) \; \forall \lambda \in (0,1) \implies y = z.$$ This is exactly the conditions that the $\lambda$-complement is unique.	
\end{proof}

\begin{remark}
	Note that all classical and quantum cases are complemented and therefore embedded into real vector spaces. It is not yet clear whether an ensemble space must be complemented or not. It is difficult to justify as it is a claim on existence of more refined ensembles. It may very well be that this is just an assumption on the decomposition, which may fail is new physical conditions.
\end{remark}

\begin{defn}
	An ensemble is \textbf{decomposable} if it is the mixture of two or more components. An ensemble space is \textbf{infinitesimally decomposable} if all ensembles are decomposable. It is \textbf{finitely decomposable} if all ensembles can be decomposed into components that are not decomposable.
\end{defn}

\begin{remark}
	The non-decomposable ensembles are exactly the extreme points of the convex space.
\end{remark}



\subsection{Physical examples}

These are the examples that we need to keep in mind, which will need to be fully characterized in terms of convex spaces and their properties.


\section{Convex subspaces}

Since the goal is to use convex spaces to create a unified theory for all types of mechanics, we need a construction that generalizes the notion of support/subspaces of ensembles.

\begin{defn}
	Let $\Ens$ be an ensemble space and $X \subseteq \Ens$ be a subset. We say that $X$ is a \textbf{subspace} of $\Ens$ if it contains all the mixtures and all the components of its elements. That is, for every $\ens_1, \ens_2, \ens_3$ and $p \in (0,1)$ such that $p\ens_1 + \bar{p} \ens_2 = \ens_3$ we have:
	\begin{itemize}
	\item $\ens_1, \ens_2 \in X$ implies $\ens_3 \in X$
	\item $\ens_3 \in X$ implies $\ens_1, \ens_2 \in X$
\end{itemize}
\end{defn}

\begin{defn}
	Let $\Ens$ be an ensemble space and $X \subseteq \Ens$.  The \textbf{convex span} of $X$, noted $\cospan(X)$, is the smallest subspace containing $X$.
\end{defn}

\begin{remark}
	As defined, the convex span of two elements will include all their possible mixtures (i.e. the segment that connects them), all possible decompositions (i.e. all lines that pass through them) plus, recursively, all other mixtures and decompositions that can be reached from those. Physically, the idea is that not all ensembles, pure states in particular, cannot be physically realized. Therefore, the inclusion of pure states in our convex space stems from a theoretical idealization useful to decompose and strudy the problem. It makes sense, then, that a subspace comes with all its idealizations.
\end{remark}

\begin{example} Classical discrete subspaces.
	Let $S$ be a set of $n$ possible discrete states and let $\Ens$ the space of probability distributions over the set $S$ (i.e. $\Ens$ is an $n$-simplex and $S$ are its extreme points). A subspace $X$ of $\Ens$ is a convex hull of a subset $U$ of $S$. That is, a subspace of $\Ens$ is the space of probability distributions over a subset of the cases. Geometrically, it is one of the sides (possibly recursively) of the simplex.
	
	To see this, first note that the convex hull $X$ of any subset $U$ of extreme points $S$ is a subspace. In fact, it will contain all convex combinations of $U$, and any element can only be decomposed in convex combinations of $U$. Second, note that only convex hulls of a subset of extreme points can be a subspace. In fact, any element of $\Ens$ can be expressed as a non-trivial convex combination of a set of extreme points $U$. Therefore, if an element is present in a subspace $X$, then $U \subset X$, which means all elements of the convex hull of $U$ are in $X$.
	
	TODO: This does not seem to generalize to the infinite case, as we do not have closure on infinite convex combinations.
\end{example}

\begin{example} Quantum subspaces.
	Let $\mathcal{H}$ be an $n$-dimensional Hilbert space and let $\Ens$ be the space of density matrices (i.e. positive semi-definite self-adjoint operators with trace one). A subspace $X$ of $\Ens$ is the space of density matrices of a subspace $U$ of $\mathcal{H}$. That is, a subspace of $\Ens$ is the space of mixed states over a subspace of pure states.
	
	To see this, first note that the space of density matrices $X$ of a subspace $U$ of $\mathcal{H}$ is a subspace of $\Ens$. In fact, $X$ it will contain all convex combinations of its elements. Moreover, any element $x \in X$ can only be decomposed in a convex combination of pure states of $U$. Therefore any convex decomposition of $x$ has all its elements in $X$. Second, note that only the space of density matrices $X$ of a subspace $U$ of $\mathcal{H}$ is a subspace of $\Ens$. In fact, any element $x$ of $\Ens$ can be expressed as a non-trivial convex combination of orthogonal pure states, its eigenstates. These elements will span a subspace $U$ of $\mathcal{H}$. From those elements, we can construct an equal mixture which represents the maximally mixed states and, mathematically, is the identity operator $I/m$ divided by the number of elements $m \leq n$ of $U$. The equal mixture of any orthogonal basis of $U$ will also give the maximally mixed state. Therefore, given an element $x$, any subspace that contain $x$ will also contain a basis of $U$, the maximally mixed state $I/n$, all possible basis of $U$, which means all the pure states, and finally all convex combinations of the pure states, which means all possible density matrices, all possible mixed states.
	
	TODO: This does not seem to generalize to the infinite case, as we do not have closure on infinite convex combinations.
\end{example}

\begin{remark}
	TODO: Ideally, the definition would generalize to the classical continuous case. Let $S$ be a symplectic manifold and let $\Ens$ be the space of probability distributions over $S$ (i.e. the space of continuous functions over $S$ that are integrable and integrate to one). We would like to have a definition for which a subspace of $\Ens$ is a set of functions whose support is a subset of an open region $U \subseteq S$. That is, a subspace contains all probability distributions defined over a subset of the cases. One problem is that continuity forces the function to go to zero on the boundary of $U$, and the speed of the convergence cannot be change with a finite convex combination. That is, the convex combination of functions that go down like an exponential will also go down like an exponential.
\end{remark}

\begin{remark}
	TODO: An open question is whether there is a relationship between an ensemble space being complemented (i.e. the embedding of the convex space into a vector space) and features of the convex span.
\end{remark}

\begin{defn}
	Two subspaces $X_1, X_2 \subseteq \Ens$ are \textbf{orthogonal} if they are disjoint. That is, $X_1 \cap X_2 = \emptyset$. Two ensembles $\ens_1, \ens_2 \in \Ens$ are \textbf{disjoint} if they belong to two orthogonal subspaces. That is, $\cospan(\ens_1) \cap \cospan(\ens_2) = \emptyset$.
\end{defn}

\begin{remark}
	Two classical distributions are disjoint if and only if their support is disjoint. Two quantum states are disjoint if and only if they are the mixture of states that belong to disjoint subspaces.
	
	Physically, two disjoint ensembles are distinct in the sense that they will never produce overlapping results.
\end{remark}

\begin{remark}
	TODO: The definitions that follows work only for finitely decomposable spaces (i.e. all extreme points are in the space). Ideally, we want definitions that works for all cases in the same way.
\end{remark}

\begin{defn}
	A \textbf{pure state} is an ensemble that is not decomposable. Given a subspaces $X \in \Ens$ be a subspace, it's \textbf{dimension}, noted $\dim(X)$, is the minimum number of pure states $x_i \in X$ such that $\cospan(x_i) = X$.
\end{defn}

\begin{remark}
	In the classical discrete case, the subspaces are the simplexes. The dimensionality of the subspace is the number of vertices of the simplex (i.e. dimensionality of the simplex plus one). That is, a single pure state is a subspace of dimension one; a line segment is a subspace of dimension two; and so on.
	
	In quantum mechanics, subspaces correspond to the projective spaces of subspaces of the complex vector space.
\end{remark}


\section{Entropy}

In both classical and quantum mechanics, the Shannon/Gibbs/von Neumann entropy defines the geometric structure of the space. Convex combinations also define geometric features, precisely because the entropy can be calculated from the probabilities, at least in the classical discrete case and quantum. On the continuum, we also know that being able to define a frame invariant entropy is what leads to requiring the symplectic structure. Moreover, the space of distributions over position and velocity is isomorphic, as a convex space, to the space of distributions over position and momentum. Therefore, it does not seem that, in the classical continuous case, the convex space structure is enough. Exactly what geometric information is and is not in the convex structure and the entropy is something that needs to be understood.

\begin{defn}
	Given the coefficients $\{p_i\} \in [0,1]$ such that $\sum p_i = 1$, the \textbf{entropy of the coefficients} is defined as $I(\{p_i\}) = - \sum p_i \log p_i $.
\end{defn}

\begin{ext}
	An ensemble space is equipped with an \textbf{entropy} function $S : X \to \mathbb{R}$ with the following properties
	\begin{itemize}
		\item \textbf{Continuity}
		\item \textbf{Strict concavity}: $S(p_1\rho_1 + p_2 \rho_2) \geq p_1 S(\rho_1) + p_2 S(\rho_2)$ with the equality holding if and only if $\rho_1 = \rho_2$
		\item \textbf{Coefficient bound}: $S(p_1\rho_1 + p_2 \rho_2) \leq I(p_1, p_2) + p_1 S(\rho_1) + p_2 S(\rho_2)$
	\end{itemize}
\end{ext}

\begin{remark}
	Note that the bounds for entropy can also be written as
	$$ 0 \leq S(p_1\rho_1 + p_2 \rho_2) - (p_1 S(\rho_1) + p_2 S(\rho_2)) \leq I(p_1, p_2).$$
	The term in the middle compares the entropy of the mixture to the average entropy of the components. It essentially tells us that the entropy can only increase during the mixture, and it can only increase up to the entropy of the coefficients.
	
	Note that if an ensemble can be decomposed in multiple ways, the decomposition such that the entropy of the weights is minimal provides the least upper bound.
	
	TODO: The Shannon entropy can be recovered in the discrete case with 3 minimal requirements. Are the above constraints enough to recover a unique entropy function (up to a constant) that matches the right entropy in the discrete/continuum/quantum cases? If not, is this just a question on requirements on entropy or also requirements on the space?
	
	TODO: Does the existence of an entropy function put a limit on the convex space? For example, a line is convex space with no extreme points. We can parameterize the space with a variable $x$ such that the mixture of two ensembles is identified by the average $x$. Entropy can be written as $S(x)$. Since $S(x)$ is strictly concave over an infinite range, it will tend to minus infinity either when $x$ tends to infinity or minus infinity. The entropy A function cannot be strictly concave 
\end{remark}

\begin{remark}
	We can define
	$$JSD(\rho_1, \rho_2) = S\left(\frac{1}{2}\rho_1 + \frac{1}{2} \rho_2\right) - \left(\frac{1}{2} S(\rho_1) + \frac{1}{2} S(\rho_2)\right).$$
	Both in the classical and quantum case, this corresponds to the Jensen-Shannon divergence \url{https://en.wikipedia.org/wiki/Jensen%E2%80%93Shannon_divergence}. It is connected to the Fisher-Rao metric \url{https://en.wikipedia.org/wiki/Fisher_information_metric#Relation_to_the_Jensen%E2%80%93Shannon_divergence} and its square root is a distance function.
	
	The question is whether the bounds above are enough to make the square root of the $JSD$ a distance function in the general case.
\end{remark}

\begin{defn}
	Two mixtures $x, y \in X$ are \textbf{disjoint} if $S(p_1\rho_1 + p_2 \rho_2) - (p_1 S(\rho_1) + p_2 S(\rho_2)) = I(p_1, p_2)$.
\end{defn}

\begin{remark}
	Two classical distributions are disjoint if and only if their support is disjoint. Two quantum states are disjoint if and only if they are the mixture of states that belong to disjoint subspaces.
	
	Physically, two disjoint ensembles are distinct in the sense that they will never produce overlapping results.
\end{remark}

\begin{prop}
	The thermodynamic entropy is an entropy over an ensemble space.
\end{prop}
\begin{justification}
	TODO This can be easily done in terms of information theory and statistical mechanics. An argument based on thermodynamics would be better.
\end{justification}

\begin{remark}
	Note that $JSD(\rho_1, \rho_2 = )S(p_1\rho_1 + p_2 \rho_2) - (p_1 S(\rho_1) + p_2 S(\rho_2))$ is the Jensen Shannon divergence (\url{https://en.wikipedia.org/wiki/Jensen%E2%80%93Shannon_divergence}). The square root is a distance function, and therefore makes the convex space a metric space.
\end{remark}

\subsection{Physical examples}

We review the physical examples in terms of the entropy.

\subsubsection{Classical case - discrete}

The entropy is given by $-\sum p_i \log p_i$ for both finite and countable cases. Therefore the entropy for each pure state is assumed to be zero. The entropy of the maximally mixed state is $\log n$ where $n$ is the number of pure states. The entropy increase as we go from pure states to the maximally mixed state. The level sets (i.e. the fibers) of the entropy form a series of concentric "shells".

TODO: Check that the entropy is finite for all convex combinations. Check that it diverges as we go to a limit point that is not in the ensemble space.

TODO: Is it important that all pure states have zero entropy?

\subsubsection{Classical case - continuous}

The entropy is given by $- \int \rho \log \rho d\mu$ where $\mu$ is the Liouville measure and $\rho$ is the probability density over canonical coordinates. If a different measure is used, or if the coordinates are not canonical, the formula gives the wrong result.

The entropy can be used to check whether two ensembles are overlapping and define the size of the support. That is: take $\rho \in X$; construct the set $\text{dis}(\rho)$ of all the ensembles that are disjoint from $\rho$; now construct the set $\text{con}(\rho)$ of all ensembles that are disjoint from all elements of $\text{dis}(\rho)$. These will be the set of ensembles that share the supports. Find the supremum of $\text{dis}(\rho)$ in terms of the entropy. If it doesn't exists, the support is infinite. If it does, the support is finite and the measure is $2^{\sup S(\text{con}(\rho))}$.

TODO: Study the shell of zero entropy states. For example, it should not be path connected: all uniform distributions with support of the same finite size (in terms of Liouville measure) will have the same entropy. The region, however, need not be contiguous. Since we cannot continuously transform a single region into two disjoint regions, there will be different distributions at zero entropy that cannot be transformed continuously.

\subsubsection{Quantum case}

In quantum mechanics, the entropy is the von Neumann entropy for both finite and infinite dimension. The entropy for the maximally mixed state is $\log n$ where $n$ is the dimensionality of the Hilbert space.

TODO: How exactly do the bounds of entropy interact with multiple decomposition?

