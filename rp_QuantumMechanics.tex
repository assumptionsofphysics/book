

\def\>{\rangle}
\def\<{\langle}

%TODO: decide whether the identity operator should be 1 or $I$ or something else

\chapter{Quantum mechanics}

% TODO: review classical uncertainty principle

Review of the mathematical formulation of quantum mechanics
* wave-function (postulates - rays of Hilbert space)
- states are rays in Hilbert space
- observables are Hermitian operator inner product, Born rule, projection as final state
- composite systems (tensor product)
- unitary evolution
* mixtures and entropy

Composite system
* Show that system composition of quantum systems that is itself a quantum system gives the tensor product
* Review the arguments in the paper
* see if using a Segre embedding would make our life easier

\section{Schroedinger equation and unitary evolution}

TODO Since time is continuous, we can understand time evolution as a continuous series of time steps $U_{dt}$. Given that the evolution is unitary, we can write the infinitesimal time step as $U_{dt} = 1 + \imath \hbar H dt$. That is the following condition
%TODO: proof?
TODOis equivalent to \ref{rp-qm-uev-condUnitaryEvolution}.


\begin{equation}\label{rp-qm-uev-condSchroedingerEq}
	\tag{DR-SCEQ}
	\eqtext{The evolution follows the equation $\imath \hbar \frac{d}{dt} \psi(t) = H \psi(t)$ where $H$ is a self-adjoint operator}
\end{equation}

\begin{equation}\label{rp-qm-uev-condUnitaryEvolution}
	\tag{DR-UNIT}
	\eqtext{The evolution is unitary: preserves the norm: $U^{\dagger}_{\Delta t} U_{\Delta t} = 1$} 
\end{equation}


Note that unitary evolution $U_{\Delta t}$ preserves the norm of state vectors. That is
\begin{equation}
	\begin{aligned}
        \<\psi(t + \Delta t) | \psi(t + \Delta t)\> &= \< U_{\Delta t} \psi(t) | U_{\Delta t} \psi(t)\> \\
        &=\<  \psi(t) | U^{\dagger}_{\Delta t} U_{\Delta t} | \psi(t)\> \\
        &=\<  \psi(t) | 1 | \psi(t)\> \\
        &=\<  \psi(t) | \psi(t)\>.
	\end{aligned}
\end{equation}
The reverse is true as well: a transformation that preserves the norm of all state vectors is a unitary operation. We have
\begin{equation}
	\begin{aligned}
		\< \psi(t+dt) | \psi(t+dt) \> &= \< U_{\Delta t} \psi(t) | U_{\Delta t} \psi(t)\>
		&= \< \psi(t) | U^{\dagger}_{\Delta t} U_{\Delta t} | \psi(t)\>.
	\end{aligned}
\end{equation}
If $\< \psi(t+dt) | \psi(t+dt) \> = \<  \psi(t) | \psi(t)\>$ for all $\psi$, then we must have $U^{\dagger}_{\Delta t} U_{\Delta t} = 1$, which is the definition of a unitary operator. Therefore condition
\begin{equation}\label{rp-qm-uev-condNormalized}
	\tag{DR-NORM}
	\eqtext{The evolution preserves the norm: $\< \psi(t) | \psi(t) \> = \< \psi(t+dt) | \psi(t+dt) \>$} 
\end{equation}
is equivalent to condition \ref{rp-qm-uev-condUnitaryEvolution}.
%TODO: check that the cross terms are also going to work (probably look at the norm of $\psi + \phi$)

We now look at the square of the inner product between two states infinitesimally close in time. We have
\begin{equation}
	\begin{aligned}
		|\<\psi(t) | \psi(t + dt)\>|^2 &= \< \psi(t) | U_{d t} \psi(t)\> \< U_{d t} \psi(t) | \psi(t)\> \\
		&=\<  \psi(t) | 1 + \frac{H dt}{\imath \hbar} | \psi(t)\> \<\psi(t) | \left(1 + \frac{H dt}{\imath \hbar} \right)^{\dagger}| \psi(t)\> \\
		&=\<  \psi(t) | 1 + \frac{H dt}{\imath \hbar} | \psi(t)\> \<\psi(t) | 1 - \frac{H dt}{\imath \hbar} | \psi(t)\> \\
		&=\< \psi(t) | \psi(t)\> \< \psi(t) | \psi(t)\> + \<  \psi(t) | \psi(t)\> \<\psi(t) | - \frac{H dt}{\imath \hbar} | \psi(t)\> \\
		&+ \< \psi(t) | \frac{H dt}{\imath \hbar} | \psi(t)\> \<\psi(t) | \psi(t)\> + O(dt^2)\\
		&=1 - \<\psi(t) | \frac{H dt}{\imath \hbar} | \psi(t)\> + \<\psi(t) | \frac{H dt}{\imath \hbar} | \psi(t)\>+ O(dt^2). \\
		&=1 + O(dt^2).
	\end{aligned}
\end{equation}
Which means that the projection of the state after an infinitesimal time step over the initial state is one. Or, equivalently, that the change of state is orthogonal to the initial state. Note that the converse is true as well. If the square of the inner product between two states infinitesimally close in time is one, then the evolution is unitary. In fact, if $U_{\Delta t}$ is continuous, we can write the infintesimal time step as $U_{d t} = 1 + \frac{O dt}{\imath \hbar}$ from some operator $O$. If we follow the previous argument in reverse, we will find that $O = O^\dagger$. That is, $O$ is self-adjoint. Therefore $U_{dt}$ is unitary by condition \ref{rp-qm-uev-condSchroedingerEq}. Therefore condition
\begin{equation}\label{rp-qm-uev-condUnitaryBorn}
	\tag{DR-UBOR}
	\eqtext{The square of the inner product between to states infinitesimally close in time is one: $|\< \psi(t) | \psi(dt) \> |^2 = 1$} 
\end{equation}
is yet another equivalent condition to unitary evolution.

The above condition makes sense if we understand what happens geometrically. A unitary evolution is effectively a rotation in the Hilbert space, and the infinitesimal change under a rotation is perpendicular to the radius. In our case, the state vector $|\psi(t)\>$ is the radius and $|d\psi(t)\>$ is the change. Therefore condition
\begin{equation}\label{rp-qm-uev-condOrthogonalChange}
	\tag{DR-ORTH}
	\eqtext{The change in time is orthogonal to the state being changed: $|\< \psi(t) | d\psi(dt) \> |^2 = 0$} 
\end{equation}
is yet another equivalent condition.

Another way to characterize unitary evolution is through what happens to an orthonormal basis $|e_i\>$. Given that a unitary evolution preserves the inner product, $\< U_{\Delta t} e_i | U_{\Delta t} e_j \> = \< e_i | e_j \> = \delta_{ij}$. Therefore the unitary evolution maps an orthonormal basis to another orthonormal basis. The converse is also true, if $U_{\Delta t}$ maps an orthonormal basis to another orthonormal basis, then the inner product is preserves. To see this, we can simply expand any vector as a superposition of the basis vector. That is
\begin{equation}
	\begin{aligned}
		\<\psi(t) | \phi(t)\> &= \< c_i e_i | d_j e_j\> \\
		&= c_i^* d_j \< e_i | e_j\> \\
		&= c_i^* d_j \< U_{\Delta t} e_i | U_{\Delta t} e_j\> \\
		&= \< U_{\Delta t} c_i e_i | U_{\Delta t} d_i e_j\> \\
		&= \< U_{\Delta t} \psi(t) | U_{\Delta t} \phi(t)\> \\
		&= \< \psi(t + \Delta t) | \phi(t + \Delta t)\> \\
	\end{aligned}
\end{equation}
Therefore condition
\begin{equation}\label{rp-qm-uev-condOrthonormalBasis}
	\tag{DR-OBAS}
	\eqtext{The change in time maps orthonormal basis to orthonormal basis: $\< U_{\Delta t} e_i | U_{\Delta t} e_j \> = \< e_i | e_j \> = \delta_{ij}$} 
\end{equation}

When looking at classical mechanics, we saw that Hamiltonian evolution implied conservation of entropy. The connection is still present in quantum mechanics. Let $\rho(t)$ be the evolution of a density operator. Let $\rho(t) = p_i |e_i(t) \> \<e_i(t)|$ the diagonalized expression for the density operator, where the basis can be, in general, different at each moment in time. Given the linearity of time evolution, the $p_i |e_i(t) \> \<e_i(t)| = p_i |U_{\Delta t}e_i(t) \> \<U_{\Delta t} e_i(t)|$. That is, the coefficients of the probability distribution over the basis remain the same. Only the basis changes. Since the entropy only depends on the coefficients $p_i$, the entropy remains unchanged.

The converse is true: if an evolution preserves entropy, then it is unitary. First of all, note that pure states are the only states with zero entropy. Therefore pure states must be mapped to pure states. Now, consider two orthogonal states. An equal mixture of the two will give us a maximally mixed state over two pure states. This means that, given linearity, a maximally mixed state over two pure states can only be mapped to another maximally mixed state of two pure states. Therefore orthogonal states must be mapped to orthogonal states. This means that an evolution that preserves entropy must map an orthonormal basis to another orthonormal basis, and therefore it is a unitary evolution. That is, condition
\begin{equation}\label{rp-qm-uev-condInfoEntropy}
	\tag{DR-INFO}
	\eqtext{The evolution preserves information entropy} 
\end{equation}
is equivalent to \ref{rp-qm-uev-condSchroedingerEq}.

Note that in quantum mechanics both information entropy and thermodynamic entropy coincide, in the sense that we do not have two different definition in quantum statistical mechanics as we have in classical statistical mechanics (i.e. logarithm of count of states and Shannon entropy). Also not that the condition is stronger in quantum mechanics, as it is enough to recover full unitary evolution. In classical mechanics it could only recover the preservation of number of states (i.e. of determinism) while it didn't recover the full Hamiltonian evolution (i.e. the conservation of independent of the DOFs).

* Orthogonal states remain orthogonal = unitary evolution
* Inner product between state and state at next time step = 1 => unitary evolution

Physical
* Conservation of entropy
* Conservation of probability (i.e. unitary evolution)
* (?) Preservation of uncertainty (would require CPTP) in terms of gaussian state/observarbles
* Continuous sequences of measurements 

Unitary evolution is deterministic and reversible evolution

\section{next}

Projections are processes with equilibria (all fine states are equilibria)
* (?) Projection is not enough: need compatility with a unitary evolution
* Show that projections cannot decrease entropy
* Eigenstates of projections are equlibria => all quantum states are equilibria of projection
  - Mathematically, this is what Hilbert spaces add on top of Banach spaces
* Analogy to thermodynamics (context is like different type of ensembles)
* Unitary evolution is quasi-static evolution (like in thermodynamics)
  - Make the parallel to S-matrix calculation where we put initial state at minus infinity, and final state at plus infinity for a process that actually last "femtoseconds"

Observables
* (?) Convex maps of mixed states are Hermitian operators
* Is it useful to note that any observable is compatible to some unitary? That is, any observable is left unchanged by a unitary?

Open quantum systems
* Review open quantum system (Lindblad master equation)
* (?) recover CPTP maps 
- Linear maps : map mixed states to mixed states while preserving mixtures
- Trace preserving : map trace one operators to trace one operators
- Positive : mixed states have non-negative eigenvalues
- Completely Positive: (?) need a characterization of completely positive in terms of only the system, without the ancilla
* (?) Kraus operator, jump operatos: how are they related? If they are?
* (?) What are the possible motions on a Bloch sphere? That is, what are the possible vector fields described by the Lindblad equation

Classical limit
* Classical mechanics is the high entropy limit of quantum mechanics
- Find classical transformations that increase entropy, show that they are all "unitaries" plus stretch of phase space
- Find equivalent of phase space stretching in quantum mechanics
- See that it is a CPTP map only defined in the anti-normal ordering
- Show that it rescales the commutator by the factor for phase space stretching
- Show that this is equivalent to the limit of $\hbar \to 0$

Negative probability in quantum mechanics
* QM on phase space (Wigner functions - Hussimi - Glauber–Sudarshan)
* (real) convex combination vs affine combination vs linear combination
* => QM on phase space is using affine combinations, since convex combinations are not sufficient

Quantum states as equibria
* Show that for a unitary evolution, eigenstates are equilibria
* Show that any quantum state is an eigenstate of some unitary
* => all quantum states are equilibria of unitary

(?) Recover spin 1/2 (two state systems)
* Space of ensembles that is fully characterized by an average direction.
  - Gaussian states are fully identified by average and standard deviation
  - Suppose we have "guassian states" of directions with same standard deviation
  - -> the space is a ball
  - (?) how much can we recover?
* Space of directional pure states
  - I have a state space for directions in space
  - Recycle the argument that we have to be able to put a frame invariant distribution over it
  - -> two sphere is the only symplectic sphere and therefor is the only space
  - (?) why are ensembles the Bloch?
  
Problems with infinite dimensional spaces
* Unsuitability of Hilbert space
  - Hilbert spaces with different number of DOF are equivalent
  - Ill-defined expectations: states with infinite energy/undefined position and so on...
  - TODO: take all the arguments from the paper
* Suitability of Schwartz spaces for special case

(?) Generalize this to arbitrary dimensions

(?) Random things to look at to see whether they are helpful
* Kähler manifold, interplay of symplectic structure with metric tensor
* Is anything of the old arguments salvageable?
* See if there is anything we can get from GPT or other reconstructions



  
