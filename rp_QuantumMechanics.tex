

\def\>{\rangle}
\def\<{\langle}

%TODO: decide whether the identity operator should be 1 or $I$ or something else

\chapter{Quantum mechanics}

% TODO: review classical uncertainty principle

Review of the mathematical formulation of quantum mechanics
* wave-function (postulates - rays of Hilbert space)
- states are rays in Hilbert space
- observables are Hermitian operator inner product, Born rule, projection as final state
- composite systems (tensor product)
- unitary evolution
* mixtures and entropy

Composite system
* Show that system composition of quantum systems that is itself a quantum system gives the tensor product
* Review the arguments in the paper
* see if using a Segre embedding would make our life easier

\section{Schroedinger equation and unitary evolution}

We start with the Schroedinger equation
\begin{equation}\label{rp-qm-uev-condSchroedingerEq}
	\tag{DR-SCEQ}
	\eqtext{The evolution follows the equation $\imath \hbar \frac{d}{dt} \psi(t) = H \psi(t)$ where $H$ is a self-adjoint operator}
\end{equation}

We now want to write the evolution as an operator that act on a wave function. The infinitesimal time step is going to be
\begin{equation}
	\begin{aligned}
		\psi(t+dt) &= \psi(t) + d\psi(t) + \psi(t) + \frac{d}{dt} \psi(t) dt \\
		&= \psi(t)+ \frac{H dt}{\imath \hbar} \psi(t)\\
		&= \left(1 + \frac{H dt}{\imath \hbar}\right)\psi(t) \\
		&= U_{dt}\psi(t)
	\end{aligned}
\end{equation}

Note that, since $H$ is a self-adjoint operator, $U_{dt}$ is a unitary operator. In fact
\begin{equation}
	\begin{aligned}
		U_{dt}^\dagger U_{dt} &= \left(1 + \frac{H dt}{\imath \hbar}\right)^\dagger \left(1 + \frac{H dt}{\imath \hbar}\right) \\
		&= \left(1 + \frac{H^\dagger dt}{(- \imath) \hbar}\right) \left(1 + \frac{H dt}{\imath \hbar}\right)\\
		&= \left(1 - \frac{H dt}{\imath \hbar}\right) \left(1 + \frac{H dt}{\imath \hbar}\right) \\
		&= 1 + \left(\frac{H dt}{\imath \hbar}\right)^2 \\
		&= 1 + O(dt^2)
	\end{aligned}
\end{equation}

This works the other way around. If we have an infinitesimal operator that is unitary, we can express it in that form. That is
\begin{equation}
	\begin{aligned}
		U_{dt}\psi(t) &= \psi(t) + \frac{d\psi(t)}{dt} dt \\
		&= (1 + A dt)\psi(t) \\
		U_{dt}^\dagger U_{dt} &= \left(1 + A dt\right)^\dagger \left(1 + A dt\right) \\
		&= \left(1 + A^\dagger dt\right) \left(1 + A dt\right)\\
		&= 1 + \left(A + A^\dagger\right)dt + A^\dagger A dt^2 \\
		&= 1 + \left(A + A^\dagger\right)dt + O(dt^2) \\
	\end{aligned}
\end{equation}
If we want the operator to be unitary, then we must have that $A = - A^\dagger $, which means $A$ is (anti-self-adjoint) (?). Which means we can write it as $A = \frac{H}{\imath \hbar}$ where $H$ is a self-adjoint operator. The evolution follows the Schroedinger equation.

This means that 
\begin{equation}\label{rp-qm-uev-condUnitaryEvolution}
	\tag{DR-UNIT}
	\eqtext{The evolution is unitary: preserves the norm: $U^{\dagger}_{\Delta t} U_{\Delta t} = 1$} 
\end{equation}
is equivalent to \ref{rp-qm-uev-condSchroedingerEq}.

Note that if the evolution is unitary, then the inner product is conserved. In fact
\begin{equation}
	\begin{aligned}
		\< U \phi | U \psi\> &= \< \phi | U^\dagger U | \psi \> \\
		&=\<  \phi | \psi \>.
	\end{aligned}
\end{equation}
The argument works in reverse as well, therefore
\begin{equation}\label{rp-qm-uev-condInnerProductPreserved}
	\tag{DR-INN}
	\eqtext{The evolution preserves the inner product : $\< U_{\Delta t} \phi | U_{\Delta t} \psi\> =\<  \phi | \psi \>$} 
\end{equation}
is equivalent to \ref{rp-qm-uev-condUnitaryEvolution}.


In particular, unitary evolution will preserve the norm of vectors, since $|psi|^2 = \<\psi | \psi \>$. The reverse is true as well: a transformation that preserves the norm of all state vectors is a unitary operation. This is because the inner product can be expressed, through the polarization identity, in terms of the norm.
\begin{equation}
	\begin{aligned}
		\< \phi | \psi \> = \frac{1}{4}\left( |\phi + \psi|^2 - |\phi - \psi|^2 - \imath |\phi + \imath \psi|^2 +\imath |\phi - \imath \psi|^2 \right)
	\end{aligned}
\end{equation}
If the norm is conserved, then, the inner product is conserved and the evolultion is unitary. Therefore condition
\begin{equation}\label{rp-qm-uev-condNormalized}
	\tag{DR-NORM}
	\eqtext{The evolution preserves the norm: $\< \psi(t) | \psi(t) \> = \< \psi(t+dt) | \psi(t+dt) \>$} 
\end{equation}
is equivalent to condition \ref{rp-qm-uev-condInnerProductPreserved}.

We now look at the square of the inner product between two states infinitesimally close in time. We have
\begin{equation}
	\begin{aligned}
		|\<\psi(t) | \psi(t + dt)\>|^2 &= \< \psi(t) | U_{d t} \psi(t)\> \< U_{d t} \psi(t) | \psi(t)\> \\
		&=\<  \psi(t) | 1 + \frac{H dt}{\imath \hbar} | \psi(t)\> \<\psi(t) | \left(1 + \frac{H dt}{\imath \hbar} \right)^{\dagger}| \psi(t)\> \\
		&=\<  \psi(t) | 1 + \frac{H dt}{\imath \hbar} | \psi(t)\> \<\psi(t) | 1 - \frac{H dt}{\imath \hbar} | \psi(t)\> \\
		&=\< \psi(t) | \psi(t)\> \< \psi(t) | \psi(t)\> + \<  \psi(t) | \psi(t)\> \<\psi(t) | - \frac{H dt}{\imath \hbar} | \psi(t)\> \\
		&+ \< \psi(t) | \frac{H dt}{\imath \hbar} | \psi(t)\> \<\psi(t) | \psi(t)\> + O(dt^2)\\
		&=1 - \<\psi(t) | \frac{H dt}{\imath \hbar} | \psi(t)\> + \<\psi(t) | \frac{H dt}{\imath \hbar} | \psi(t)\>+ O(dt^2). \\
		&=1 + O(dt^2).
	\end{aligned}
\end{equation}
Which means that the projection of the state after an infinitesimal time step over the initial state is one. Note that the converse is true as well. If the square of the inner product between two states infinitesimally close in time is one, then the evolution is unitary. In fact, if $U_{\Delta t}$ is continuous, we can write the infintesimal time step as $U_{d t} = 1 + \frac{O dt}{\imath \hbar}$ from some operator $O$. If we follow the previous argument in reverse, we will find that $O = O^\dagger$. That is, $O$ is self-adjoint. Therefore $U_{dt}$ is unitary by condition \ref{rp-qm-uev-condSchroedingerEq}. Therefore condition
\begin{equation}\label{rp-qm-uev-condUnitaryBorn}
	\tag{DR-UBOR}
	\eqtext{The square of the inner product between to states infinitesimally close in time is one: $|\< \psi(t) | \psi(t + dt) \> |^2 = 1$} 
\end{equation}
is yet another equivalent condition to unitary evolution.

The above condition makes sense if we understand what happens geometrically. A unitary evolution is effectively a rotation in the Hilbert space, and the infinitesimal change under a rotation is perpendicular to the radius. In our case, the state vector $|\psi(t)\>$ is the radius and $|d\psi(t)\>$ is the change. Therefore condition
\begin{equation}\label{rp-qm-uev-condOrthogonalChange}
	\tag{DR-ORTH}
	\eqtext{The change in time is orthogonal to the state being changed: $|\< \psi(t) | d\psi(dt) \> |^2 = 0$} 
\end{equation}
is yet another equivalent condition.

Another way to characterize unitary evolution is through what happens to an orthonormal basis $|e_i\>$. Given that a unitary evolution preserves the inner product, $\< U_{d t} e_i | U_{d t} e_j \> = \< e_i | e_j \> = \delta_{ij}$. Therefore the unitary evolution maps an orthonormal basis to another orthonormal basis. The converse is also true, if $U_{\Delta t}$ maps an orthonormal basis to another orthonormal basis, then the inner product is preserves. To see this, we can simply expand any vector as a superposition of the basis vector. That is
\begin{equation}
	\begin{aligned}
		\<\psi(t) | \phi(t)\> &= \< c_i e_i | d_j e_j\> \\
		&= c_i^* d_j \< e_i | e_j\> \\
		&= c_i^* d_j \< U_{d t} e_i | U_{d t} e_j\> \\
		&= \< U_{d t} c_i e_i | U_{d t} d_i e_j\> \\
		&= \< U_{d t} \psi(t) | U_{d t} \phi(t)\> \\
		&= \< \psi(t + d t) | \phi(t + d t)\> \\
	\end{aligned}
\end{equation}
Therefore condition
\begin{equation}\label{rp-qm-uev-condOrthonormalBasis}
	\tag{DR-OBAS}
	\eqtext{The change in time maps orthonormal basis to orthonormal basis: $\< U_{d t} e_i | U_{d t} e_j \> = \< e_i | e_j \> = \delta_{ij}$} 
\end{equation}

\subsection{Physical conditions}

In classical mechanics we saw that Hamiltonian evolution was equivalent to determinism and reversibility. The same applies to quantum mechanics. If an evolution is deterministic and reversible, then knowing the initial state means knowing the final state and vice-versa. Given that quantum mechanics is a probabilistic theory, this translates as being able to predict the final state or reconstruct the initial state in terms of probability. That is, if I know the initial state with 100\% probability, I will know the final state with 100\% probability and vice-versa. Mathematically, this means that if we start with a pure state $\rho(t) = |\psi(t) \> \< \psi(t)|$, then the final state will have to be $\rho(t+\Delta t) = |\psi(t+\Delta t) \> \< \psi(t+\Delta t)|$. Given that both $\rho(t)$ and $\rho(t + \Delta t)$ are trace one operators, the norm of both $\psi(t)$ and $\psi(t+\Delta t)$ must be unitary. That is, a deterministic and reversible evolution must preserve the norm. Which means that 
\begin{equation}\label{rp-qm-uev-condDetRev}
	\tag{DR-EV}
	\eqtext{The evolution is deterministic and reversible} 
\end{equation}

The same argument can be developed on probability distributions. Suppose that we have a mixed state, meaning that it can be understood as a classical mixture of a set of orthogonal states. If we have a deterministic and reversibile evolution evolution, we will be able to map the same distribution to another set of final state such that the probability remains unchanged. That is, if we are able to write $\rho(t) = p_i |e_i(t) \> \<e_i(t)|$, we must have $\rho(t+ \Delta t) = p_i |U_{\Delta t}e_i(t) \> \<U_{\Delta t} e_i(t)|$. For that two work, the final set of states $U_{\Delta t}e_i(t)$ must be orthogonal to each, therefore if the process transports probability distributions from initial to final observers, then it must map orthonormal basis to orthonormal basis. This means the evolution is unitary and 
\begin{equation}\label{rp-qm-uev-condProbTrans}
	\tag{DR-EV}
	\eqtext{The evolution preserves probability distributions} 
\end{equation}
is another equivalent conditions.

The idea here is that, in the end, determinism and reversibility is effectively a permutation of objects, while keeping the closeness (i.e. the differentiable topology) and distance (i.e. the inner product) the same.


When looking at classical mechanics, we saw that determinism and reversibility, at a conceptual level, is equivalent to conservation of information entropy. This connection is still valid in quantum mechanics. Let $\rho(t)$ be the evolution of a density operator. Let $\rho(t) = p_i |e_i(t) \> \<e_i(t)|$ the diagonalized expression for the density operator, where the basis can be, in general, different at each moment in time. Given the linearity of time evolution, the evolution will be $U_{dt} \left( p_i |e_i(t) \> \<e_i(t)| \right)= p_i |U_{\Delta t}e_i(t) \> \<U_{\Delta t} e_i(t)|$. That is, the coefficients of the probability distribution over the basis remain the same. Only the basis changes. Since the entropy only depends on the coefficients $p_i$, the entropy remains unchanged.

The converse is true: if an evolution preserves entropy, then it is unitary. First of all, note that pure states are the only states with zero entropy. Therefore pure states must be mapped to pure states. Now, consider two orthogonal states. An equal mixture of the two will give us a maximally mixed state over two pure states. This means that, given linearity, a maximally mixed state over two pure states can only be mapped to another maximally mixed state of two pure states. Therefore orthogonal states must be mapped to orthogonal states. This means that an evolution that preserves entropy must map an orthonormal basis to another orthonormal basis, and therefore it is a unitary evolution. That is, condition
\begin{equation}\label{rp-qm-uev-condInfoEntropy}
	\tag{DR-INFO}
	\eqtext{The evolution preserves information entropy} 
\end{equation}
is equivalent to \ref{rp-qm-uev-condSchroedingerEq}.

Note that in quantum mechanics both information entropy and thermodynamic entropy coincide, in the sense that we do not have two different definition in quantum statistical mechanics as we have in classical statistical mechanics (i.e. logarithm of count of states and Shannon entropy). What we can show, however, is that unitary evolution can be understood as an infinite sequence of projections that perturb the system minimally.

Suppose that you prepare a spin 1/2 system in the horizontal direction and then measure the vertical direction. You have equal chance of measuring spin up or spin down. Now suppose you put a polarizer (?) at a 45 degree angle in between. There will be two projection, first on the 45 degree angle and then along the vertical direction. The chance of ending in the up state is greater than ending the in down state. You can imagine inserting other to (polarizers (?)) in between again, making 4 proections at a 22.5 degree angle between each pair. The chance of ending up in the up state is even greater. We can imagine, then, that if we put infinitely many steps at infinitesimal angles, the chance of being in the up state at the end will be 100\%.

We saw, in fact, that for a unitary evolution $\<\psi(t+dt) | \psi(t)\> =1$, that is the projection of the state at a future time step on the previous time step is one. This can be understood as making a projective measurement on an observable that is slightly different to one for which $\psi(t)$ is an eigenstate. That is, we can understand unitary evolution as an infiniteimal sequence of projections at each time step.

Another way of understanding this is that determinism and reversibility can be used for both measuring and preparing states. That is, if we prepare a system in a given state, we can use unitary evolution to prepare a system in the future state. Conversely, if we can measure a system in a given state, we can use unitary evolution to infer the state of the system at a prior time. Therefore, we can understand determinism and reversibility as a series of preparations or measurements. Since measurements in quantum mechanics are projections, it makes sense that we can understand unitary evolution as a sequence of projections. Therefore 
\begin{equation}\label{rp-qm-uev-condProjectionSequence}
	\tag{DR-PSEQ}
	\eqtext{The evolution is an sequence of measurements at each time step} 
\end{equation}
is equivalent to \ref{rp-qm-uev-condDetRev}.

\section{next}

Projections are processes with equilibria (all fine states are equilibria)
* (?) Projection is not enough: need compatility with a unitary evolution
* Show that projections cannot decrease entropy
* Eigenstates of projections are equlibria => all quantum states are equilibria of projection
  - Mathematically, this is what Hilbert spaces add on top of Banach spaces
* Analogy to thermodynamics (context is like different type of ensembles)
* Unitary evolution is quasi-static evolution (like in thermodynamics)
  - Make the parallel to S-matrix calculation where we put initial state at minus infinity, and final state at plus infinity for a process that actually last "femtoseconds"

Observables
* (?) Convex maps of mixed states are Hermitian operators
* Is it useful to note that any observable is compatible to some unitary? That is, any observable is left unchanged by a unitary?

Open quantum systems
* Review open quantum system (Lindblad master equation)
* (?) recover CPTP maps 
- Linear maps : map mixed states to mixed states while preserving mixtures
- Trace preserving : map trace one operators to trace one operators
- Positive : mixed states have non-negative eigenvalues
- Completely Positive: (?) need a characterization of completely positive in terms of only the system, without the ancilla
* (?) Kraus operator, jump operatos: how are they related? If they are?
* (?) What are the possible motions on a Bloch sphere? That is, what are the possible vector fields described by the Lindblad equation

Classical limit
* Classical mechanics is the high entropy limit of quantum mechanics
- Find classical transformations that increase entropy, show that they are all "unitaries" plus stretch of phase space
- Find equivalent of phase space stretching in quantum mechanics
- See that it is a CPTP map only defined in the anti-normal ordering
- Show that it rescales the commutator by the factor for phase space stretching
- Show that this is equivalent to the limit of $\hbar \to 0$

Negative probability in quantum mechanics
* QM on phase space (Wigner functions - Hussimi - Glauber–Sudarshan)
* (real) convex combination vs affine combination vs linear combination
* => QM on phase space is using affine combinations, since convex combinations are not sufficient

Quantum states as equibria
* Show that for a unitary evolution, eigenstates are equilibria
* Show that any quantum state is an eigenstate of some unitary
* => all quantum states are equilibria of unitary

(?) Recover spin 1/2 (two state systems)
* Space of ensembles that is fully characterized by an average direction.
  - Gaussian states are fully identified by average and standard deviation
  - Suppose we have "guassian states" of directions with same standard deviation
  - -> the space is a ball
  - (?) how much can we recover?
* Space of directional pure states
  - I have a state space for directions in space
  - Recycle the argument that we have to be able to put a frame invariant distribution over it
  - -> two sphere is the only symplectic sphere and therefor is the only space
  - (?) why are ensembles the Bloch?
  
Problems with infinite dimensional spaces
* Unsuitability of Hilbert space
  - Hilbert spaces with different number of DOF are equivalent
  - Ill-defined expectations: states with infinite energy/undefined position and so on...
  - TODO: take all the arguments from the paper
* Suitability of Schwartz spaces for special case

(?) Generalize this to arbitrary dimensions

(?) Random things to look at to see whether they are helpful
* Kähler manifold, interplay of symplectic structure with metric tensor
* Is anything of the old arguments salvageable?
* See if there is anything we can get from GPT or other reconstructions



  
